\documentclass[11pt]{article}
\usepackage{amsmath,amssymb,float,amsfonts,enumitem,tcolorbox}

%!TEX root = EM_singular_models.tex

% PDF margin etc settings
\setlength{\textwidth}{\paperwidth}
\addtolength{\textwidth}{-6cm}
\setlength{\textheight}{\paperheight}
\addtolength{\textheight}{-4cm}
\addtolength{\textheight}{-1.1\headheight}
\addtolength{\textheight}{-\headsep}
\addtolength{\textheight}{-\footskip}
\setlength{\oddsidemargin}{0.5cm}
\setlength{\evensidemargin}{0.5cm}

\long\def\comment#1{}
\definecolor{battleshipgrey}{rgb}{0.52, 0.52, 0.51}
\definecolor{darkgray}{rgb}{0.66, 0.66, 0.66}
\definecolor{darkgreen}{rgb}{0.0, 0.2, 0.13}
\definecolor{darkspringgreen}{rgb}{0.09, 0.45, 0.27}
\definecolor{dukeblue}{rgb}{0.0, 0.0, 0.61}
\definecolor{olivedrab7}{rgb}{0.24, 0.2, 0.12}
\definecolor{darkblue}{rgb}{0.0, 0.0, 0.55}
\definecolor{darkscarlet}{rgb}{0.34, 0.01, 0.1}
\definecolor{candyapplered}{rgb}{1.0, 0.03, 0.0}
\definecolor{ao(english)}{rgb}{0.0, 0.5, 0.0}
\definecolor{applegreen}{rgb}{0.55, 0.71, 0.0}
\definecolor{orange}{rgb}{1.0, 0.65, 0.0}

\newcommand{\red}[1]{\textcolor{red}{#1}}
\newcommand{\blue}[1]{\textcolor{blue}{#1}}
\newcommand{\green}[1]{\textcolor{green}{#1}}
\newcommand{\orange}[1]{\textcolor{orange}{#1}}
\newcommand{\highlight}[1]{\textcolor{applegreen}{#1}}

\newcommand{\rdcomment}[1]{{\bf{{\red{{RD --- #1}}}}}}
\newcommand{\yccomment}[1]{{\bf{{\orange{{YC --- #1}}}}}}

\renewcommand{\vec}[1]{\mathbf{#1}}
\newcommand{\mat}[1]{\mathbf{#1}}

% Basic Math notations
% Dimension etc.
\newcommand{\dims}{\ensuremath{d}}
\newcommand{\real}{\ensuremath{\mathbb{R}}}
\newcommand{\naturalnum}{\ensuremath{\mathbb{N}}}
\newcommand{\Ind}{\ensuremath{\mathbb{I}}}
% Probability
\newcommand{\borel}{\ensuremath{\mathcal{B}}}
\newcommand{\lebesgue}{\ensuremath{\mathfrak{L}}}
\newcommand{\powerset}{\ensuremath{\mathfrak{P}}}
\newcommand{\lebesguemeasure}{\ensuremath{\lambda}}
\newcommand{\ball}{\ensuremath{\mathbb{B}}}
\newcommand{\Exs}{\ensuremath{{\mathbb{E}}}}
\newcommand{\Prob}{\ensuremath{{\mathbb{P}}}}
\newcommand{\Law}{\mathcal{L}}
\newcommand{\Normal}{\ensuremath{\mathcal{N}}}
\newcommand{\Ber}{\ensuremath{\mbox{Ber}}}
\newcommand{\Ent}{\text{Ent}}
\newcommand{\Var}{\text{Var}}


% Brackets
\newcommand{\brackets}[1]{\left[ #1 \right]}
\newcommand{\parenth}[1]{\left( #1 \right)}
\newcommand{\bigparenth}[1]{\big( #1 \big)}
\newcommand{\biggparenth}[1]{\bigg( #1 \bigg)}
\newcommand{\braces}[1]{\left\{ #1 \right \}}
\newcommand{\abss}[1]{\left| #1 \right |}
\newcommand{\angles}[1]{\left\langle #1 \right \rangle}
\newcommand{\ceils}[1]{\left\lceil #1 \right \rceil}
\newcommand{\floors}[1]{\left\lfloor #1 \right \rfloor}
\newcommand{\tp}{^\top}

% Some vector/matrix norms
\newcommand{\matnorm}[3]{|\!|\!| #1 | \! | \!|_{{#2}, {#3}}}
\newcommand{\matsnorm}[2]{\left|\!\left|\!\left| #1 \right|\!\right|\!\right|_{{#2}}}
\newcommand{\vecnorm}[2]{\left\| #1\right\|_{#2}}


\title{STAT 154: Homework 3}
\author{Release date: \textbf{Thursday, February 21}}
\date{Due by: \textbf{11 PM, Wednesday, Mar 6}}

\begin{document}

\maketitle

This homework follows regular submission format, i.e., it is to be
submitted by each student in the class \emph{individually}--no teams!

\section{A closer look at EM (25 points)}

In this question, we consider a simple mixture  model and work our way through
a derivation of the EM updates.\footnote{The question borrows ideas
from one of the homeworks of machine learning class CS 189, Spring 2018.}
\emph{While the question looks long, please be patient in reading it--the
description itself will help you understand EM better and also different
parts of the problem only appear long because of the detailed explanation.}

We work with the following simple two mixture model:
\begin{align}
  Z &\sim \text{Bernoulli}(1-w)+1\notag\\
  X\vert Z=1 &\sim \mathcal{N}(\mu_1, 1), \quad \text{and}\notag\\
  X\vert Z=2 &\sim \mathcal{N}(\mu_2, 1),
  \label{eq:model}
\end{align}
where $Z$ denotes the label of the Gaussian from which $X$ is drawn.
Given a set of observations only for $X$ (i.e., the labels are unobserved), 
our goal is to infer the maximum-likelihood parameters for $\mu_1, \mu_2
$ and $w$. Note that to simplify your calculations, we have fixed the variance
parameter and assumed it to be known.


\begin{enumerate}[label=(\alph*)] 
\item (3 points) Let $\vec{\theta} =(\mu_1, \mu_2, w)$ denote the parameters
of the
model.
\textbf{Write down the expressions of the joint likelihood
$p(X=x, Z=1; \vec{\theta})$ and $p(X=x, Z=2; \vec{\theta})$.
What is the marginal likelihood $p(X=x; \vec{\theta})$ and the log-likelihood
$\ell(X=x; \vec{\theta})$?
Given $n$ i.i.d. samples $\braces{x_1, \ldots, x_n}$, write the expression
for the log-likelihood $\ell(X_1=x_1, \ldots, X_n=x_n; \vec{\theta})$.}

\item   (4 points)
To simplify notation, from now on, we use the notation 
\begin{align*}
 \ell(x; \vec{\theta}) &= \ell({X} = {x}; \vec{\theta}),
 \quad\text{and}
 \quad p(x, k; \vec\theta) = p(X=x, Z=k; \vec\theta).
\end{align*}
Let $q$ denote a distribution on the (hidden) labels $\{Z_i\}_{i=1}^n$ given by
\begin{align}
  q(Z_1 = z_1, \ldots, Z_n = z_n) = \prod_{i=1}^n q_i(Z_i=z_i). \label{eq:q}
\end{align}
Note that since $Z \in\{1, 2\}$, $q$ has $n$ parameters, namely $\{q_i(Z_i=1), i=1, \ldots, n\}$.
{\bf Show that for a given point $x_i$, we have }
\begin{align}
  \ell(x_i; \vec{\theta})
  &\geq \mathcal{F}_i(\vec{\theta}; q_i)
  :=\underbrace{\sum_{k=1}^2 q_i(k) \log p (x_i, k; \vec{\theta})}_{\mathcal{L}(x_i; \vec\theta, q_i)} + \underbrace{\sum_{k=1}^2 {q}_i(k) \log\left(\frac{1}{{q}_i(k)}\right)}_{H(q_i)}, 
  \label{eq:lower_bound}
\end{align}
where $H({q}_i)$ denotes the Shannon-entropy of the distribution
$q_i$. 
Thus {\bf conclude that we obtain the following lower bound on the log-likelihood:}
\begin{align}
  \ell(\{x_i\}_{i=1}^n; \vec\theta)  \geq 
  \mathcal{F}(\vec{\theta}; q) := \sum_{i=1}^n\mathcal{F}_i(\vec{\theta}; {q}_i).
  \label{eq:F_defn}
\end{align}
% Note that, as explained before, the two parameters for our case, we have 
% $\vec\theta=(\mu_1, \sigma_1, \mu_2, \sigma_2)$ and $q$ has $n$ parameters
% since we have $n$ data points and $Z$ takes only two values.

{\em Hint: Jensen's inequality, the concave-$\cap$ nature of the log,
  and reviewing lecture notes might be useful.} 

\item  (2 points) The EM algorithm can be considered a
coordinate-ascent\footnote{A coordinate-ascent algorithm is just one
  that fixes some coordinates and maximizes the function with respect
  to the others as a way of taking iterative improvement steps.} algorithm
  on the
lower bound $\mathcal{F}(\vec{\theta}; q)$ derived in the previous part,
where we ascend with respect to $\vec\theta$ and $q$ in an alternating fashion.
More precisely, one iteration of the EM algorithm is made up of 2-steps:
\begin{align*}
  q^{t+1} &= \arg\max_{q} \mathcal{F}(\vec{\theta}^t; q) \quad&\text{(E-step)}\\
  {\vec{\theta}}^{t+1} &\in \arg\max_{\vec\theta} \mathcal{F}(\vec{\theta}; q^{t+1}).  \quad&\text{(M-step)}
\end{align*}
Given an estimate $\vec\theta^t$, the previous part tells us that 
$\ell(\{x_i\}_{i=1}^n; \vec{\theta}^t)  \geq 
  \mathcal{F}(\vec{\theta}^t; q)$. {\bf Verify that equality holds in
    this bound if we plug in $q(Z_1=z_1, \ldots, Z_n=z_n) 
  = \prod_{i=1}^n p(Z = z_i\vert X= x_i; \vec{\theta}^t)$
and hence we can conclude that 
\begin{align}
\label{eq:estep}
  q^{t+1}(Z_1=z_1, \ldots, Z_n=z_n) 
  &= \prod_{i=1}^n p(Z = z_i\vert X= x_i; \vec{\theta}^t).
\end{align}
is a valid maximizer for the problem $\max_{q} \mathcal{F}(\vec{\theta}^t; q)$ 
and hence a valid E-step update.}

\item  (2 points) \textbf{Derive the expressions for $p(Z = 1\vert X= x_i;
\vec{\theta}^t)$
and $p(Z = 2\vert X= x_i; \vec{\theta}^t)$ to complete the E-step computations
where $\vec\theta^t = (\mu_1^t, \mu_2^t, w^t)$.}

\item (3 points) 
We now discuss the M-step. Using the definitions from equations~\eqref{eq:lower_bound} and \eqref{eq:F_defn}, we have that
\begin{align*}
  \mathcal{F}(\vec{\theta}; q^{t+1}) = 
  \sum_{i=1}^n(\mathcal{L}(x_i; \vec\theta, q_i^{t+1})
    + H(q_i))
  = H(q^{t+1}) + \sum_{i=1}^n\mathcal{L}(\vec x_i; \vec\theta, q_i^{t+1}),
\end{align*}
where we have used the fact that entropy in this case is given by $H(q^{t+1}) = \sum_{i=1}^nH(q_i^{t+1})$.
Notice that although (as computed in previous part), $q^{t+1}$ depends
on $\vec\theta^t$, the M-step only involves maximizing $\mathcal{F}(\vec{\theta}; q^{t+1})$
with respect to just the parameter $\vec\theta$ while keeping the 
parameter $q^{t+1}$ fixed.
Now, noting that the entropy term $H(q^{t+1})$ does not depend on 
the parameter $\vec\theta$, we conclude that the M-step simplifies to solving for
\begin{align*}
  \arg\max_{\vec\theta} \underbrace{\sum_{i=1}^n\mathcal{L}(\vec x_i; \vec\theta, q_i^{t+1})}_{=:\mathcal{L}(\vec{\theta}; q^{t+1})} .
\end{align*}
We use the simplified notation
\begin{align*}
  q^{t+1}_i:= q_i^{t+1}(Z_i=1) \quad\text{ and } \quad  
  1-q^{t+1}_i:= q_i^{t+1}(Z_i=2)
\end{align*}
and recall that $\vec \theta = (\mu_1, \mu_2, w)$.

{\bf Show that the expression for $\mathcal{L}(\vec{\theta}; q^{t+1})$ for the $2$-mixture case is given by
\begin{align*}
  &\mathcal{L}((\mu_1, \mu_2, w); q^{t+1}) \\
  % &= \sum_{i=1}^n (q^{t+1}_i \log p(x_i, 1) + (1-q_i^{t+1})\log p(x_i, 2))\\
  % &= \sum_{i=1}^n (q^{t+1}_i \log 0.5* p(x_i\vert Z_i=1) + (1-q_i^{t+1})\log 0.5* p(x_i\vert Z_i =2))\\
  % &=  - n\log 2 -\sum_{i=1}^n \left[q^{t+1}_i  \left(\frac{(x_i-\mu_1)^2}{2\sigma_1^2}+\frac{1}{2}\log{2w}+\log \sigma_1\right) + (1-q_i^{t+1}) \left(\frac{(x_i-\mu_2)^2}{2\sigma_2^2}+\frac{1}{2}\log{2w}+\log \sigma_2\right)\right]\\
  &\quad\quad=  C + \sum_{i=1}^n\left[q^{t+1}_i  \left(\log w-
  \frac{(x_i-\mu_1)^2}{2}\right) + (1-q_i^{t+1}) 
  \left(\log(1-w)-\frac{(x_i-\mu_2)^2}{2}\right)\right],
\end{align*}
where $C$ is a constant that does not depend on $\vec\theta$ or $q^{t+1}$.}

\item (6 points) Using the expression of $\mathcal{L}$ from the previous
part, {\bf derive the expressions for the gradients of $\mathcal{L}
(\vec{\theta}; q^{t+1})$ with respect to $\mu_1, \mu_2, w$.
By setting these gradients to zero, show that the M-step updates are given
by}
\begin{align*}
  \mu_1^{t+1} &= \frac{\sum_{i=1}^n q_i^{t+1} x_i}{\sum_{i=1}^n q_i^{t+1}},\quad
  \mu_2^{t+1} = \frac{\sum_{i=1}^n (1-q_i^{t+1}) x_i}{\sum_{i=1}^n (1-q_i^
  {t+1})},\quad\text{and}\quad
  w^{t+1} = \frac{\sum_{i=1}^n q_i^{t+1}}{n}.
\end{align*}
This complete the derivation of EM updates in the simpler model introduced
in the beginning of the problem.

\item (5 points) We now see EM in action and compare it with K-means. (No
code is
required in the submission.)
Generate $1000$ samples from the following mixture of two Gaussians in two
dimensions:
\begin{align}
  Z &= \text{Bernoulli}(0.5) + 1 \notag\\
  X|Z=1 &\sim \mathcal{N}\left( \begin{bmatrix}0\\ 0\end{bmatrix}, \begin{bmatrix}1 & 0\\ 0 & 1\end{bmatrix}\right )\notag\\
  X|Z=2 &\sim \mathcal{N}\left( \begin{bmatrix}1\\ 0\end{bmatrix}, \begin{bmatrix}1 & 0\\ 0 & 4\end{bmatrix}\right )
  \label{eq:model_simulate}
\end{align}
where $I_2$ denotes the identity matrix in two dimensions.
\textbf{Generate both $(Z, X)$ for the data and scatter plot the $X$ values
with color based on $Z$.}
\textbf{Run K-means on $X$ with $K=2$ and report the cluster centers
and scatter plot the data with estimated labels.
Run EM on $X$ to fit two clusters too (use Mclust package with $G=2$)
and report the mean parameters
and scatter plot the data with estimated labels.
Note that we need $3$ plots for this part.
Justify qualitatively why the cluster centers obtained by K-means and
EM and the estimated label are different.}

\end{enumerate}
\end{document}