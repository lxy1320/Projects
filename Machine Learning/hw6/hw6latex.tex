\documentclass[11pt]{article}\usepackage[]{graphicx}\usepackage[]{color}
%% maxwidth is the original width if it is less than linewidth
%% otherwise use linewidth (to make sure the graphics do not exceed the margin)
\makeatletter
\def\maxwidth{ %
  \ifdim\Gin@nat@width>\linewidth
    \linewidth
  \else
    \Gin@nat@width
  \fi
}
\makeatother

\definecolor{fgcolor}{rgb}{0.345, 0.345, 0.345}
\newcommand{\hlnum}[1]{\textcolor[rgb]{0.686,0.059,0.569}{#1}}%
\newcommand{\hlstr}[1]{\textcolor[rgb]{0.192,0.494,0.8}{#1}}%
\newcommand{\hlcom}[1]{\textcolor[rgb]{0.678,0.584,0.686}{\textit{#1}}}%
\newcommand{\hlopt}[1]{\textcolor[rgb]{0,0,0}{#1}}%
\newcommand{\hlstd}[1]{\textcolor[rgb]{0.345,0.345,0.345}{#1}}%
\newcommand{\hlkwa}[1]{\textcolor[rgb]{0.161,0.373,0.58}{\textbf{#1}}}%
\newcommand{\hlkwb}[1]{\textcolor[rgb]{0.69,0.353,0.396}{#1}}%
\newcommand{\hlkwc}[1]{\textcolor[rgb]{0.333,0.667,0.333}{#1}}%
\newcommand{\hlkwd}[1]{\textcolor[rgb]{0.737,0.353,0.396}{\textbf{#1}}}%
\let\hlipl\hlkwb

\usepackage{framed}
\makeatletter
\newenvironment{kframe}{%
 \def\at@end@of@kframe{}%
 \ifinner\ifhmode%
  \def\at@end@of@kframe{\end{minipage}}%
  \begin{minipage}{\columnwidth}%
 \fi\fi%
 \def\FrameCommand##1{\hskip\@totalleftmargin \hskip-\fboxsep
 \colorbox{shadecolor}{##1}\hskip-\fboxsep
     % There is no \\@totalrightmargin, so:
     \hskip-\linewidth \hskip-\@totalleftmargin \hskip\columnwidth}%
 \MakeFramed {\advance\hsize-\width
   \@totalleftmargin\z@ \linewidth\hsize
   \@setminipage}}%
 {\par\unskip\endMakeFramed%
 \at@end@of@kframe}
\makeatother

\definecolor{shadecolor}{rgb}{.97, .97, .97}
\definecolor{messagecolor}{rgb}{0, 0, 0}
\definecolor{warningcolor}{rgb}{1, 0, 1}
\definecolor{errorcolor}{rgb}{1, 0, 0}
\newenvironment{knitrout}{}{} % an empty environment to be redefined in TeX

\usepackage{alltt}
\usepackage{amsmath,amssymb,float,amsfonts,enumitem,tcolorbox,bm}

\usepackage{algorithm}
\usepackage[colorlinks=True,linkcolor=magenta,citecolor=blue,urlcolor=blue,pagebackref=true,backref=true]{hyperref}
\usepackage[noend]{algpseudocode}

\include{macros}



\title{STAT 154: Homework 6}
\author{Release date: \textbf{Sunday, April 7}}
\date{Due by: \textbf{11 PM, Sunday, April 21}}
\IfFileExists{upquote.sty}{\usepackage{upquote}}{}
\begin{document}

\maketitle

\section*{The honor code}

\begin{enumerate}[label=(\alph*)]
\item Please state the names of people who you worked with for this homework.
You can also provide your comments about the homework here.
\begin{tcolorbox}
\vspace{20mm}
\end{tcolorbox}

\item  Please type/write the following sentences yourself and sign at the end. We want to make
it \emph{extra} clear that nobody cheats even unintentionally.

\emph{I hereby state that all of my solutions were entirely in my words and were written by me.
I have not looked at another student’s solutions and I have fairly credited all external 
sources in this write up.}
\begin{tcolorbox}
\vspace{20mm}
\end{tcolorbox}

\end{enumerate}

\section*{Submission instructions}

\begin{itemize}
\item It is a good idea to revisit your notes, slides and reading;
and synthesize their main points BEFORE doing the
homework.

\item No .Rnw file is provided. You may use templates from previous homeworks
if you want.

\item \textbf{For the parts that ask you to 
implement/run some R code, your answer
should look something like this (code followed by result):}
\begin{knitrout}
\definecolor{shadecolor}{rgb}{0.969, 0.969, 0.969}\color{fgcolor}\begin{kframe}
\begin{alltt}
\hlstd{myfun}\hlkwb{<-} \hlkwa{function}\hlstd{()\{}
\hlkwd{show}\hlstd{(}\hlstr{'this is a dummy function'}\hlstd{)}
\hlstd{\}}
\hlkwd{myfun}\hlstd{()}
\end{alltt}
\begin{verbatim}
## [1] "this is a dummy function"
\end{verbatim}
\end{kframe}
\end{knitrout}
Note that this is automatically generated if you use
the R sweave environment.

\item You need to submit the following:
\begin{enumerate}
\item A pdf of your write-up to ``HW6 write-up'' that includes code for
Problem 4.
\item No \emph{separate} code submission is required for this HW. You have
to include the code in your submission for Problem 4 in the write-up itself.
\end{enumerate}
\item Ensure a proper submission to gradescope, otherwise it will
not be graded. \textbf{This time we will not entertain any regrade
requests for improper submission.}
\end{itemize}

% \newpage
\section*{Homework Overview}
This homework covers kernel ridge regression and classification. The first
problems attempts to make you comfortable with computational complexity
related questions.
\newpage
\section{Computational complexity \small{(10 pts)}} % (fold)
\label{sec:computational_complexity}

Read about the big-O notation for the wiki page \url{https://en.wikipedia.org/wiki/Big_O_notation}
and then answer the following using the big-O notation.

  In the following questions, by computational complexity we refer
  to the number of addition/multiplication operations between two real numbers.
  To elaborate, we assume that two real numbers or multiplying them takes
  unit operations.
  Adding $k$ real numbers has $k$ computational complexity.
  Computing the multiplication of $k$ pairs of numbers would have $k$ 
  computational complexity as well.\\

  Now let $\vec a, \vec v \in \real^d$ and $\mat A, \mat B \in \real^{n
  \times
  d}$ and answer the following questions:
\begin{enumerate}
  \item (2 pts) What is the computational complexity of computing $\vec
  a + \vec
  v$? What is the computational complexity of computing $\vec a \tp \vec
  v$?
  \item (2 pts) What is the computational complexity of computing the matrix
  $\mat A + \mat B$?  How much space does storing the matrix $\mat A$ require? 
  \item (4 pts) What is the computational complexity of computing the
  vector $\mat
  A \vec v$? How about computing the matrix $\mat A \tp \mat B$?
  \item (2 pts) What is the complexity of computing the vector $\mat A \tp
  \mat B \vec v$? \\
  {\emph{Hint: There are two ways to do it, one is naive and one is smart.
  You are encouraged to think and report both of them in your answer.}}
\end{enumerate}
% section computational_complexity (end)

\section{Kernel Methods \small(23 pts)} % (fold)
\label{sec:kernel_methods}
For the following problems, you can assume that inverting a $p \times p$
matrix takes order $p^3$ operations, i.e., the computational complexity
of matrix inversion of size $p$ is $O(p^3)$.
Also for this problem, we use slightly different notation for dimensions
in order to remain consistent with the note and the lecture.
\begin{enumerate}
  \item (2 pts) 
  Let $\mat X \in \real^{n \times \ell}$ and $\vec y \in \real^n$.
  Recall
  that the ridge estimate for the problem
  \begin{align}
    \min_{\theta \in \real^\ell} \enorm{\mat X \theta - \vec y}^2 + \lambda
    \enorm{\theta}^2
    \label{eq:ridge}
  \end{align}
  is given by $\widehat{\theta}_\lambda = (\mat X \tp \mat X + \lambda \mat
  I_\ell)^{-1}\mat X\tp \vec y$. What is the computational complexity of
  computing
  this estimate?\\
  \emph{Hint: You may use answers from previous question.}
  \item (2 pts) Show that $\widehat{\theta}_\lambda = \mat X\tp(\mat X \mat
  X \tp + \lambda \mat I_n)^{-1} \vec y$ is also a valid estimate for the
  ridge problem~\ref{eq:ridge}.
  What is the complexity of computing this estimate?
  \item (2 pts) Compare and contrast the computational complexities from
  the previous
  two parts.
  \newcommand{\newtheta}{\widetilde{\theta}}
  \item (2 pts) Suppose we modify the problem~\eqref{eq:ridge} using a feature
  map
$\phi: \real^\ell \to \real^d$  as follows:
  \begin{align}
    &\min_{\newtheta \in \real^d} \enorm{\mat \Phi \newtheta - \vec y}^2
    + \lambda \Vert{\newtheta}\Vert_2^2
    \label{eq:ridgenew}\\
    \text{where}\quad
    &\mat \Phi = \begin{bmatrix}
      \text{---}\phi(\vec x_1)\tp \text{---}\\
      \vdots\\
      \text{---}\phi(\vec x_n)\tp \text{---}
    \end{bmatrix}\in \real^{n \times d}.
  \end{align}
  Can you provide a few reasons why we may want to do this? (Usually $d
  \geq \ell$ when we extend the $\vec x$ vectors in this fashion.)
  \item (8 pts) As discussed in class, often the choice of $\phi$ is (implicitly
  or explicitly) made such that
  \begin{align}
  \label{eq:kernel}
    \phi(\vec x)\tp\phi(\vec z) = k(x, z)
  \end{align}
  where $k:\real^\ell \times \real^\ell \to \real$ denotes the kernel function
  (that satisfies some nice properties). Can you compute the kernel functions
  for the following feature maps:
  \begin{itemize}
    \item[(a)] $\phi(x) = [x_1x_2, \frac{x_1^2}{\sqrt{2}},
    \frac{x_2^2}{\sqrt{2}}]\tp$ where $\vec x \in \real^2$ with $\vec x
    = \begin{bmatrix} x_1 \\ x_2 \end{bmatrix}$.
    \item[(b)] $\phi(x) = [1, x, \frac{x^2}{\sqrt{2!}}, \frac{x^3}{\sqrt{3!}},
    \ldots]$ where $x \in \real$.
  \end{itemize}
  On the reverse side can you compute the feature map for the following
  kernel functions (it may not be possible in some cases): 
  \begin{enumerate}
    \item[(c)] $k(x, z) = (1 + \vec x \tp \vec z)^2 + \vec x \tp \vec z$ for $\vec x, \vec z \in \real^2$.
    \item[(d)] $k(x, z) = e^{-(x - z)^2}$ for $x, z \in \real$.
  \end{enumerate}
  \item (2 pts) What are the two different ways of computing the solution
  to the
  problem~\eqref{eq:ridgenew}? 
  \item (5 pts) We now compare the complexity of the two estimates for a
  polynomial
  kernel. Compare and contrast the computational complexity of the two 
  estimates when $\vec x \in \real^\ell$ and the feature map $\phi$ is chosen
  such that the corresponding kernel function~\eqref{eq:kernel} is given
  by
  \begin{align*}
    k(\vec x, \vec z) = (1+ \vec x\tp \vec z)^p.
  \end{align*}
  Discuss when is an estimate better than the other (on computational grounds).
\end{enumerate}

% section kernel_methods (end)

\section{LDA and linear regression \small(10 pts)}
ESLII book \url{https://web.stanford.edu/~hastie/Papers/ESLII.pdf}: Question
4.2 (all 5 parts)

\section{Applied problem for classification \small(7 pts)} % (fold)
\label{sec:applied_problem_for_classification}
ISL Book: 4.11 (all 7 parts) \textbf{Show code in the write-up pdf for all
parts.}


% section applied_problem_for_classification (end)

\end{document}